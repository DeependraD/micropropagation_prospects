\PassOptionsToPackage{unicode=true}{hyperref} % options for packages loaded elsewhere
\PassOptionsToPackage{hyphens}{url}
%
\documentclass[
  man]{apa6}
\usepackage{lmodern}
\usepackage{amssymb,amsmath}
\usepackage{ifxetex,ifluatex}
\ifnum 0\ifxetex 1\fi\ifluatex 1\fi=0 % if pdftex
  \usepackage[T1]{fontenc}
  \usepackage[utf8]{inputenc}
  \usepackage{textcomp} % provides euro and other symbols
\else % if luatex or xelatex
  \usepackage{unicode-math}
  \defaultfontfeatures{Scale=MatchLowercase}
  \defaultfontfeatures[\rmfamily]{Ligatures=TeX,Scale=1}
\fi
% use upquote if available, for straight quotes in verbatim environments
\IfFileExists{upquote.sty}{\usepackage{upquote}}{}
\IfFileExists{microtype.sty}{% use microtype if available
  \usepackage[]{microtype}
  \UseMicrotypeSet[protrusion]{basicmath} % disable protrusion for tt fonts
}{}
\makeatletter
\@ifundefined{KOMAClassName}{% if non-KOMA class
  \IfFileExists{parskip.sty}{%
    \usepackage{parskip}
  }{% else
    \setlength{\parindent}{0pt}
    \setlength{\parskip}{6pt plus 2pt minus 1pt}}
}{% if KOMA class
  \KOMAoptions{parskip=half}}
\makeatother
\usepackage{xcolor}
\IfFileExists{xurl.sty}{\usepackage{xurl}}{} % add URL line breaks if available
\IfFileExists{bookmark.sty}{\usepackage{bookmark}}{\usepackage{hyperref}}
\hypersetup{
  pdftitle={Micropropagation of olive: An enterprise prospect review for Nepal},
  pdfauthor={Deependra Dhakal~\& Samita Paudel},
  pdfkeywords={keywords},
  pdfborder={0 0 0},
  breaklinks=true}
\urlstyle{same}  % don't use monospace font for urls
\usepackage{graphicx,grffile}
\makeatletter
\def\maxwidth{\ifdim\Gin@nat@width>\linewidth\linewidth\else\Gin@nat@width\fi}
\def\maxheight{\ifdim\Gin@nat@height>\textheight\textheight\else\Gin@nat@height\fi}
\makeatother
% Scale images if necessary, so that they will not overflow the page
% margins by default, and it is still possible to overwrite the defaults
% using explicit options in \includegraphics[width, height, ...]{}
\setkeys{Gin}{width=\maxwidth,height=\maxheight,keepaspectratio}
\setlength{\emergencystretch}{3em}  % prevent overfull lines
\providecommand{\tightlist}{%
  \setlength{\itemsep}{0pt}\setlength{\parskip}{0pt}}
\setcounter{secnumdepth}{-2}
% Redefines (sub)paragraphs to behave more like sections
\ifx\paragraph\undefined\else
  \let\oldparagraph\paragraph
  \renewcommand{\paragraph}[1]{\oldparagraph{#1}\mbox{}}
\fi
\ifx\subparagraph\undefined\else
  \let\oldsubparagraph\subparagraph
  \renewcommand{\subparagraph}[1]{\oldsubparagraph{#1}\mbox{}}
\fi

% set default figure placement to htbp
\makeatletter
\def\fps@figure{htbp}
\makeatother

\shorttitle{Micropropagation of olive}
\affiliation{
\vspace{0.5cm}
\textsuperscript{1} Gokuleshwor Agriculture and Animal Science College\\\textsuperscript{2} Institute of Agriculture and Animal Science}
\keywords{keywords\newline\indent Word count: X}
\usepackage{csquotes}
\usepackage{upgreek}
\captionsetup{font=singlespacing,justification=justified}

\usepackage{longtable}
\usepackage{lscape}
\usepackage{multirow}
\usepackage{tabularx}
\usepackage[flushleft]{threeparttable}
\usepackage{threeparttablex}

\newenvironment{lltable}{\begin{landscape}\begin{center}\begin{ThreePartTable}}{\end{ThreePartTable}\end{center}\end{landscape}}

\makeatletter
\newcommand\LastLTentrywidth{1em}
\newlength\longtablewidth
\setlength{\longtablewidth}{1in}
\newcommand{\getlongtablewidth}{\begingroup \ifcsname LT@\roman{LT@tables}\endcsname \global\longtablewidth=0pt \renewcommand{\LT@entry}[2]{\global\advance\longtablewidth by ##2\relax\gdef\LastLTentrywidth{##2}}\@nameuse{LT@\roman{LT@tables}} \fi \endgroup}


\DeclareDelayedFloatFlavor{ThreePartTable}{table}
\DeclareDelayedFloatFlavor{lltable}{table}
\DeclareDelayedFloatFlavor*{longtable}{table}
\makeatletter
\renewcommand{\efloat@iwrite}[1]{\immediate\expandafter\protected@write\csname efloat@post#1\endcsname{}}
\makeatother
\usepackage{lineno}

\linenumbers

\title{Micropropagation of olive: An enterprise prospect review for Nepal}
\author{Deependra Dhakal\textsuperscript{1}~\& Samita Paudel\textsuperscript{2}}
\date{}

\authornote{Author 1 is a faculty associtated to Department of Plant Breeding and Genetics, GAASC, Tribhuwan University

Author 2 is an MS (Agricultural Economics) candidate.

Correspondence concerning this article should be addressed to Deependra Dhakal, Gokuleshwor, Baitadi. E-mail: \href{mailto:ddhakal.rookie@gmail.com}{\nolinkurl{ddhakal.rookie@gmail.com}}}

\abstract{
One or two sentences providing a \textbf{basic introduction} to the field, comprehensible to a scientist in any discipline.

Two to three sentences of \textbf{more detailed background}, comprehensible to scientists in related disciplines.

One sentence clearly stating the \textbf{general problem} being addressed by this particular study.

One sentence summarizing the main result (with the words ``\textbf{here we show}'' or their equivalent).

Two or three sentences explaining what the \textbf{main result} reveals in direct comparison to what was thought to be the case previously, or how the main result adds to previous knowledge.

One or two sentences to put the results into a more \textbf{general context}.

Two or three sentences to provide a \textbf{broader perspective}, readily comprehensible to a scientist in any discipline.


}

\begin{document}
\maketitle

\hypertarget{introduction}{%
\section{Introduction}\label{introduction}}

Micropropagation technique is effective for rapid multiplication of a wide variety of crops. Popularly tissue cultured crops include, but is not limited to banana ((Banerjee \& Langhe, 1985, pp. Strosse, Vanden Houwe, andPanis (2004), Wong(1986))), strawberry ((Jones, Waller, \& Beech, 1988, pp. Passey, Barrett, and James (2003))), apple ((Lane, 1978)), potato ((Roca, Espinoza, Roca, \& Bryan, 1978)), citrus fruits ((Bitters, Murashigi, Rangan, \& Nauer, 1972), Bhat, Chitralekha, \& Chandel (1992)).

\hypertarget{private-sector-tissue-culture-initiatives-in-nepal}{%
\subsection{Private sector tissue culture initiatives in Nepal}\label{private-sector-tissue-culture-initiatives-in-nepal}}

Botanical Enterprises Pvt. Ltd.
- In vitro propagation of many species of orchid, potato, fodder, Chrysanthemum, Gerbera, African violet, Lily etc.
- Exports directly to The Netherlands.

Nepal Biotech Nursery
- Produce banana, orchid, and ornamental plants by tissue culture and non-sterile sand rooting technique.

Research Laboratory for Agriculture Biotechnology and Biochemistry (RALBB)
- Tissue culture propagation for pine, Artocarpus, Brassica.
- Anther culture of cold tolerant rice
- Modest facilities for DNA work by PCR technology and enzyme analysis.
- Facilities used for research and teaching.

Green Research and Technology (GREAT)
- Developing virus testing and elimination facilities on horticultural crops such as potato, citrus, banana, cardamom, strawberry, and some ornamental plants using tissue culture techniques.
- Has modest screen-house facility for indexing against citrus greening disease.

\hypertarget{economics-of-tissue-culture-facility}{%
\subsection{Economics of tissue culture facility}\label{economics-of-tissue-culture-facility}}

For a project profile of with an annual production capacity of 10,000, accounting for all costs (fixed and variable), it is estimated that individual seedlings' should be priced at NRs 55-60, to cover capital and operating expenses each year.

Capital expenditure comprise of laboratory room block construction, along with its holding, electrification and drainage unit, machinery, equipments, tools and shade house and miscellaneous fixed asset. Variable costs are laboratory reagents, disposables, plant material stock, utilities, consumables and marketing etc.

Similarly, accounting for survial rate of plantlets (which is estimated to be around 90\%), the returns are estimated to be more consistent than that with field propagated crops.

\hypertarget{methods}{%
\section{Methods}\label{methods}}

Tissue culture operation is undertaken in controlled environment with accurately coordinated temporal activities. Most of these activities are adapted for the protocols defined elsewhere while maintaining foundational aspect of a culturing experiment. A general guideline, providing procedure and background on the topic is given by George et al. (2008a) .Some of the ideas surrounding basic set-up of a tissue culturing facility and plant tissue manipulation for regeneration of plantlets is described herein.

It is to be marked that the scale of operation determines to a large extent the exact quantity and sophistication of instruments/equipments. Naturally, a small scale trialing facility cannot operate as efficiently as a large operating firm. This directly affects the unit costs of outputs (essentially the regenerated plantlets). Therefore, except when conducting an optimization experiment, it is recommended that tissue culture be run in a large facility in a optimized routine.

For academic institutions, a simple set-up accomodating all culturing apparatus as well as operational activities can be conducted in a laboratory room. This facility should ideally be secluded from other block, so as to check contamination.

\hypertarget{material}{%
\subsection{Material}\label{material}}

Following apparatuses are required for the preparation of tissue culture:

\begin{enumerate}
\def\labelenumi{\arabic{enumi}.}
\tightlist
\item
  Forceps (small, long and extra long)
\item
  Scalpel and scalpel blades (small, long and extra long)
\item
  Stereo microscopes
\item
  Table mount lamps/ halogen capsule bulbs
\item
  Glass slides and coverslips
\item
  Vernier calliper
\item
  Petri dishes with cover (100 mm)
\item
  Scissors (Secateur, locking type and normal)
\item
  Filter papers
\item
  Beaker (500 ml and 200ml)
\item
  Erlenmayer flask 250 ml (3-4)
\item
  Test tubes (50 ml or 100 ml) and holding platform
\item
  Pipette
\item
  Disposable pipette tips
\item
  Microbox or bottle (500 ml and 1000 ml)
\item
  Cotton roll
\item
  Tissue paper (dry)
\item
  Gloves (Nitrile or latex)
\item
  Ethanol or isopropyl alcohol (200 ml)
\item
  Detergent
\item
  Plastic tubs: 4 (for bathing vessels and storing cleaned vessels)
\item
  Hard nylon brush: 3-5 (for cleaning vessels)
\item
  pH meter
\item
  pH buffer
\item
  Autoclaving trays
\item
  Tween-20 (mild detergent for surface cleaning of explants)
\item
  Incubation chamber
\end{enumerate}

\hypertarget{procedure}{%
\subsection{Procedure}\label{procedure}}

The steps involved in any tissue culture operation are outlined below:

\begin{enumerate}
\def\labelenumi{\arabic{enumi}.}
\tightlist
\item
  Preparation of instruments and culture medium
\item
  Sterilization of instruments and culture medium
\item
  Preparation of explant
\item
  Inoculation of explant
\item
  Incubation for growth
\item
  Acclimatization of plantlets
\end{enumerate}

After obtaining the instruments, sterilize them. A newspaper may be used to wrap the instruments around before leaving them for autoclaving.

Clean and sterile culture vessels should be obtained after each run of culture by:

\begin{enumerate}
\def\labelenumi{\arabic{enumi}.}
\tightlist
\item
  First steamed for about 30-45 minutes (using autoclave)
\item
  Immerse in a pool of chromic acid for 16 hours
\item
  Rinse with water in a separate pool to wash off chromic acid
\item
  Then clean the vessal with detergent solution using nylon brush
\item
  Clean the detergent in running water
\item
  Oven dry the vessels at 60-80 degree celcius.
\end{enumerate}

Preparation of media

George et al. (2008b) has reported with extensive details on macro- and micro- element constitution of tissue culture media.

\begin{enumerate}
\def\labelenumi{\arabic{enumi}.}
\tightlist
\item
  Macro elements
\item
  Micro elements
\item
  Vitamins
\item
  Amino acids
\item
  Sucrose (Source of carbohydrate)
\item
  Deionized water
\item
  Agar
\end{enumerate}

For measuring solvents and media elements (mostly mineral salts), graduated cylinder for liquids and precise (milligram scale) measuring balances are required.

After mixing the salts in deionized water and dissolving them, the pH needs to be maintained in solution. Plants best obtain the nutrients in a pH range between 5.6 to 5.8.

The nutrient solution is prepared in Erlenmayer flasks or in test tubes (100 ml). Finally mixing of nutrient solution (100ml) and agar agar (8gm) in flask gives a nutrient media.

The flasks containing nutrient media, together with other equipments are then autoclaved (in 120 degree celcius at 15 pascal pressure)

\hypertarget{data-analysis}{%
\subsection{Data analysis}\label{data-analysis}}

We used R (Version 3.6.0; R Core Team, 2019) and the R-packages \emph{binb} (Version 0.0.5; Eddelbuettel, Zahn, \& Hyndman, 2019), \emph{knitr} (Version 1.26; Xie, 2019), \emph{papaja} (Version 0.1.0.9842; Aust \& Barth, 2018), and \emph{tidyverse} (Version 1.3.0; Wickham, 2019) for all our analyses.

\hypertarget{results}{%
\section{Results}\label{results}}

\hypertarget{discussion}{%
\section{Discussion}\label{discussion}}

\newpage

\hypertarget{conclusion}{%
\section{Conclusion}\label{conclusion}}

\hypertarget{acknowledgement}{%
\section{Acknowledgement}\label{acknowledgement}}

I am thankful to the Chairperson of college, Mr.~Santosh Joshi, for inspiring thoughts on topic and, my wife Samita for encouraging me throughout the writing process.

\hypertarget{references}{%
\section{References}\label{references}}

\begingroup
\setlength{\parindent}{-0.5in}
\setlength{\leftskip}{0.5in}

\hypertarget{refs}{}
\leavevmode\hypertarget{ref-R-papaja}{}%
Aust, F., \& Barth, M. (2018). \emph{papaja: Create APA manuscripts with R Markdown}. Retrieved from \url{https://github.com/crsh/papaja}

\leavevmode\hypertarget{ref-banerjee1985tissue}{}%
Banerjee, N., \& Langhe, E. de. (1985). A tissue culture technique for rapid clonal propagation and storage under minimal growth conditions of musa (banana and plantain). \emph{Plant Cell Reports}, \emph{4}(6), 351--354.

\leavevmode\hypertarget{ref-bhat1992regeneration}{}%
Bhat, S., Chitralekha, P., \& Chandel, K. (1992). Regeneration of plants from long-term root culture of lime, citrus aurantifolia (christm.) swing. \emph{Plant Cell, Tissue and Organ Culture}, \emph{29}(1), 19--25.

\leavevmode\hypertarget{ref-bitters1972investigations}{}%
Bitters, W., Murashigi, T., Rangan, T., \& Nauer, E. (1972). Investigations on establishing virus-free citrus plants through tissue culture. In \emph{International organization of citrus virologists conference proceedings (1957-2010)} (Vol. 5).

\leavevmode\hypertarget{ref-R-binb}{}%
Eddelbuettel, D., Zahn, I., \& Hyndman, R. (2019). \emph{Binb: 'Binb' is not 'beamer'}. Retrieved from \url{https://CRAN.R-project.org/package=binb}

\leavevmode\hypertarget{ref-george2008plant}{}%
George, E. F., Hall, M. A., \& De Klerk, G.-J. (2008a). Plant tissue culture procedure-background. In \emph{Plant propagation by tissue culture} (pp. 1--28). Springer.

\leavevmode\hypertarget{ref-george2008components}{}%
George, E. F., Hall, M. A., \& De Klerk, G.-J. (2008b). The components of plant tissue culture media i: Macro-and micro-nutrients. In \emph{Plant propagation by tissue culture} (pp. 65--113). Springer.

\leavevmode\hypertarget{ref-jones1988production}{}%
Jones, O., Waller, B. J., \& Beech, M. (1988). The production of strawberry plants from callus cultures. \emph{Plant Cell, Tissue and Organ Culture}, \emph{12}(3), 235--241.

\leavevmode\hypertarget{ref-lane1978regeneration}{}%
Lane, W. D. (1978). Regeneration of apple plants from shoot meristem-tips. \emph{Plant Science Letters}, \emph{13}(3), 281--285.

\leavevmode\hypertarget{ref-passey2003adventitious}{}%
Passey, A., Barrett, K., \& James, D. (2003). Adventitious shoot regeneration from seven commercial strawberry cultivars (fragaria\(\times\) ananassa duch.) using a range of explant types. \emph{Plant Cell Reports}, \emph{21}(5), 397--401.

\leavevmode\hypertarget{ref-R-base}{}%
R Core Team. (2019). \emph{R: A language and environment for statistical computing}. Vienna, Austria: R Foundation for Statistical Computing. Retrieved from \url{https://www.R-project.org/}

\leavevmode\hypertarget{ref-roca1978tissue}{}%
Roca, W. M., Espinoza, N., Roca, M., \& Bryan, J. (1978). A tissue culture method for the rapid propagation of potatoes. \emph{American Potato Journal}, \emph{55}(12), 691--701.

\leavevmode\hypertarget{ref-strosse2004banana}{}%
Strosse, H., Van den Houwe, I., \& Panis, B. (2004). Banana cell and tissue culture-review. \emph{Banana Improvement: Cellular, Molecular Biology, and Induced Mutations, Science Publishers, Inc., Enfield, NH, USA}, 1--12.

\leavevmode\hypertarget{ref-R-tidyverse}{}%
Wickham, H. (2019). \emph{Tidyverse: Easily install and load the 'tidyverse'}. Retrieved from \url{https://CRAN.R-project.org/package=tidyverse}

\leavevmode\hypertarget{ref-wong1986vitro}{}%
Wong, W. (1986). In vitro propagation of banana (musa spp.): Initiation, proliferation and development of shoot-tip cultures on defined media. \emph{Plant Cell, Tissue and Organ Culture}, \emph{6}(2), 159--166.

\leavevmode\hypertarget{ref-R-knitr}{}%
Xie, Y. (2019). \emph{Knitr: A general-purpose package for dynamic report generation in r}. Retrieved from \url{https://CRAN.R-project.org/package=knitr}

\leavevmode\hypertarget{ref-R-papaja}{}%
Aust, F., \& Barth, M. (2018). \emph{papaja: Create APA manuscripts with R Markdown}. Retrieved from \url{https://github.com/crsh/papaja}

\leavevmode\hypertarget{ref-banerjee1985tissue}{}%
Banerjee, N., \& Langhe, E. de. (1985). A tissue culture technique for rapid clonal propagation and storage under minimal growth conditions of musa (banana and plantain). \emph{Plant Cell Reports}, \emph{4}(6), 351--354.

\leavevmode\hypertarget{ref-bhat1992regeneration}{}%
Bhat, S., Chitralekha, P., \& Chandel, K. (1992). Regeneration of plants from long-term root culture of lime, citrus aurantifolia (christm.) swing. \emph{Plant Cell, Tissue and Organ Culture}, \emph{29}(1), 19--25.

\leavevmode\hypertarget{ref-bitters1972investigations}{}%
Bitters, W., Murashigi, T., Rangan, T., \& Nauer, E. (1972). Investigations on establishing virus-free citrus plants through tissue culture. In \emph{International organization of citrus virologists conference proceedings (1957-2010)} (Vol. 5).

\leavevmode\hypertarget{ref-R-binb}{}%
Eddelbuettel, D., Zahn, I., \& Hyndman, R. (2019). \emph{Binb: 'Binb' is not 'beamer'}. Retrieved from \url{https://CRAN.R-project.org/package=binb}

\leavevmode\hypertarget{ref-george2008plant}{}%
George, E. F., Hall, M. A., \& De Klerk, G.-J. (2008a). Plant tissue culture procedure-background. In \emph{Plant propagation by tissue culture} (pp. 1--28). Springer.

\leavevmode\hypertarget{ref-george2008components}{}%
George, E. F., Hall, M. A., \& De Klerk, G.-J. (2008b). The components of plant tissue culture media i: Macro-and micro-nutrients. In \emph{Plant propagation by tissue culture} (pp. 65--113). Springer.

\leavevmode\hypertarget{ref-jones1988production}{}%
Jones, O., Waller, B. J., \& Beech, M. (1988). The production of strawberry plants from callus cultures. \emph{Plant Cell, Tissue and Organ Culture}, \emph{12}(3), 235--241.

\leavevmode\hypertarget{ref-lane1978regeneration}{}%
Lane, W. D. (1978). Regeneration of apple plants from shoot meristem-tips. \emph{Plant Science Letters}, \emph{13}(3), 281--285.

\leavevmode\hypertarget{ref-passey2003adventitious}{}%
Passey, A., Barrett, K., \& James, D. (2003). Adventitious shoot regeneration from seven commercial strawberry cultivars (fragaria\(\times\) ananassa duch.) using a range of explant types. \emph{Plant Cell Reports}, \emph{21}(5), 397--401.

\leavevmode\hypertarget{ref-R-base}{}%
R Core Team. (2019). \emph{R: A language and environment for statistical computing}. Vienna, Austria: R Foundation for Statistical Computing. Retrieved from \url{https://www.R-project.org/}

\leavevmode\hypertarget{ref-roca1978tissue}{}%
Roca, W. M., Espinoza, N., Roca, M., \& Bryan, J. (1978). A tissue culture method for the rapid propagation of potatoes. \emph{American Potato Journal}, \emph{55}(12), 691--701.

\leavevmode\hypertarget{ref-strosse2004banana}{}%
Strosse, H., Van den Houwe, I., \& Panis, B. (2004). Banana cell and tissue culture-review. \emph{Banana Improvement: Cellular, Molecular Biology, and Induced Mutations, Science Publishers, Inc., Enfield, NH, USA}, 1--12.

\leavevmode\hypertarget{ref-R-tidyverse}{}%
Wickham, H. (2019). \emph{Tidyverse: Easily install and load the 'tidyverse'}. Retrieved from \url{https://CRAN.R-project.org/package=tidyverse}

\leavevmode\hypertarget{ref-wong1986vitro}{}%
Wong, W. (1986). In vitro propagation of banana (musa spp.): Initiation, proliferation and development of shoot-tip cultures on defined media. \emph{Plant Cell, Tissue and Organ Culture}, \emph{6}(2), 159--166.

\leavevmode\hypertarget{ref-R-knitr}{}%
Xie, Y. (2019). \emph{Knitr: A general-purpose package for dynamic report generation in r}. Retrieved from \url{https://CRAN.R-project.org/package=knitr}

\endgroup

\end{document}
